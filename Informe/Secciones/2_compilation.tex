
Para el manejo del proyecto se optó por utilizar la herramienta \texttt{make} por sobre la compilación manual, dado que \texttt{make} evalua si los archivos fuente fueron modificados después de la compilación y sólo ejecuta los scripts necesarios. Además mantiene al proyecto ordenado y facilita su desarrollo. A continuación se presenta el archivo \texttt{Makefile} que se encuentra en el directorio que contiene todos los archivos fuente y se ejecuta desde la terminal de UNIX simplemente insertando el comando \texttt{make} o en su defecto \texttt{make all}. 

	\lstinputlisting[language=make]{Makefile}

